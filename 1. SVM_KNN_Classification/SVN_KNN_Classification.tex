% Options for packages loaded elsewhere
\PassOptionsToPackage{unicode}{hyperref}
\PassOptionsToPackage{hyphens}{url}
%
\documentclass[
]{article}
\usepackage{lmodern}
\usepackage{amssymb,amsmath}
\usepackage{ifxetex,ifluatex}
\ifnum 0\ifxetex 1\fi\ifluatex 1\fi=0 % if pdftex
  \usepackage[T1]{fontenc}
  \usepackage[utf8]{inputenc}
  \usepackage{textcomp} % provide euro and other symbols
\else % if luatex or xetex
  \usepackage{unicode-math}
  \defaultfontfeatures{Scale=MatchLowercase}
  \defaultfontfeatures[\rmfamily]{Ligatures=TeX,Scale=1}
\fi
% Use upquote if available, for straight quotes in verbatim environments
\IfFileExists{upquote.sty}{\usepackage{upquote}}{}
\IfFileExists{microtype.sty}{% use microtype if available
  \usepackage[]{microtype}
  \UseMicrotypeSet[protrusion]{basicmath} % disable protrusion for tt fonts
}{}
\makeatletter
\@ifundefined{KOMAClassName}{% if non-KOMA class
  \IfFileExists{parskip.sty}{%
    \usepackage{parskip}
  }{% else
    \setlength{\parindent}{0pt}
    \setlength{\parskip}{6pt plus 2pt minus 1pt}}
}{% if KOMA class
  \KOMAoptions{parskip=half}}
\makeatother
\usepackage{xcolor}
\IfFileExists{xurl.sty}{\usepackage{xurl}}{} % add URL line breaks if available
\IfFileExists{bookmark.sty}{\usepackage{bookmark}}{\usepackage{hyperref}}
\hypersetup{
  pdftitle={Week 1 Assignment - SVM \& KNN Classification},
  pdfauthor={Omer Farooq (EDx ID: mfarooq4)},
  hidelinks,
  pdfcreator={LaTeX via pandoc}}
\urlstyle{same} % disable monospaced font for URLs
\usepackage[margin=1in]{geometry}
\usepackage{color}
\usepackage{fancyvrb}
\newcommand{\VerbBar}{|}
\newcommand{\VERB}{\Verb[commandchars=\\\{\}]}
\DefineVerbatimEnvironment{Highlighting}{Verbatim}{commandchars=\\\{\}}
% Add ',fontsize=\small' for more characters per line
\usepackage{framed}
\definecolor{shadecolor}{RGB}{248,248,248}
\newenvironment{Shaded}{\begin{snugshade}}{\end{snugshade}}
\newcommand{\AlertTok}[1]{\textcolor[rgb]{0.94,0.16,0.16}{#1}}
\newcommand{\AnnotationTok}[1]{\textcolor[rgb]{0.56,0.35,0.01}{\textbf{\textit{#1}}}}
\newcommand{\AttributeTok}[1]{\textcolor[rgb]{0.77,0.63,0.00}{#1}}
\newcommand{\BaseNTok}[1]{\textcolor[rgb]{0.00,0.00,0.81}{#1}}
\newcommand{\BuiltInTok}[1]{#1}
\newcommand{\CharTok}[1]{\textcolor[rgb]{0.31,0.60,0.02}{#1}}
\newcommand{\CommentTok}[1]{\textcolor[rgb]{0.56,0.35,0.01}{\textit{#1}}}
\newcommand{\CommentVarTok}[1]{\textcolor[rgb]{0.56,0.35,0.01}{\textbf{\textit{#1}}}}
\newcommand{\ConstantTok}[1]{\textcolor[rgb]{0.00,0.00,0.00}{#1}}
\newcommand{\ControlFlowTok}[1]{\textcolor[rgb]{0.13,0.29,0.53}{\textbf{#1}}}
\newcommand{\DataTypeTok}[1]{\textcolor[rgb]{0.13,0.29,0.53}{#1}}
\newcommand{\DecValTok}[1]{\textcolor[rgb]{0.00,0.00,0.81}{#1}}
\newcommand{\DocumentationTok}[1]{\textcolor[rgb]{0.56,0.35,0.01}{\textbf{\textit{#1}}}}
\newcommand{\ErrorTok}[1]{\textcolor[rgb]{0.64,0.00,0.00}{\textbf{#1}}}
\newcommand{\ExtensionTok}[1]{#1}
\newcommand{\FloatTok}[1]{\textcolor[rgb]{0.00,0.00,0.81}{#1}}
\newcommand{\FunctionTok}[1]{\textcolor[rgb]{0.00,0.00,0.00}{#1}}
\newcommand{\ImportTok}[1]{#1}
\newcommand{\InformationTok}[1]{\textcolor[rgb]{0.56,0.35,0.01}{\textbf{\textit{#1}}}}
\newcommand{\KeywordTok}[1]{\textcolor[rgb]{0.13,0.29,0.53}{\textbf{#1}}}
\newcommand{\NormalTok}[1]{#1}
\newcommand{\OperatorTok}[1]{\textcolor[rgb]{0.81,0.36,0.00}{\textbf{#1}}}
\newcommand{\OtherTok}[1]{\textcolor[rgb]{0.56,0.35,0.01}{#1}}
\newcommand{\PreprocessorTok}[1]{\textcolor[rgb]{0.56,0.35,0.01}{\textit{#1}}}
\newcommand{\RegionMarkerTok}[1]{#1}
\newcommand{\SpecialCharTok}[1]{\textcolor[rgb]{0.00,0.00,0.00}{#1}}
\newcommand{\SpecialStringTok}[1]{\textcolor[rgb]{0.31,0.60,0.02}{#1}}
\newcommand{\StringTok}[1]{\textcolor[rgb]{0.31,0.60,0.02}{#1}}
\newcommand{\VariableTok}[1]{\textcolor[rgb]{0.00,0.00,0.00}{#1}}
\newcommand{\VerbatimStringTok}[1]{\textcolor[rgb]{0.31,0.60,0.02}{#1}}
\newcommand{\WarningTok}[1]{\textcolor[rgb]{0.56,0.35,0.01}{\textbf{\textit{#1}}}}
\usepackage{graphicx,grffile}
\makeatletter
\def\maxwidth{\ifdim\Gin@nat@width>\linewidth\linewidth\else\Gin@nat@width\fi}
\def\maxheight{\ifdim\Gin@nat@height>\textheight\textheight\else\Gin@nat@height\fi}
\makeatother
% Scale images if necessary, so that they will not overflow the page
% margins by default, and it is still possible to overwrite the defaults
% using explicit options in \includegraphics[width, height, ...]{}
\setkeys{Gin}{width=\maxwidth,height=\maxheight,keepaspectratio}
% Set default figure placement to htbp
\makeatletter
\def\fps@figure{htbp}
\makeatother
\setlength{\emergencystretch}{3em} % prevent overfull lines
\providecommand{\tightlist}{%
  \setlength{\itemsep}{0pt}\setlength{\parskip}{0pt}}
\setcounter{secnumdepth}{-\maxdimen} % remove section numbering

\title{Week 1 Assignment - SVM \& KNN Classification}
\author{Omer Farooq (EDx ID: mfarooq4)}
\date{1/12/2020}

\begin{document}
\maketitle

\hypertarget{question-2.1}{%
\section{QUESTION 2.1}\label{question-2.1}}

\textbf{Describe a situation or problem from your job, everyday life,
current events, etc., for which a classification model would be
appropriate. List some (up to 5) predictors that you might use.}

At T-Mobile HQ in Seattle area where I am based, my team helps get
analytics products built for our procurement and supply chain teams. One
of the key insights we provide to our procurement teams is the
visibility into company's spend data. We ingest transaction level
invoices data from Accounts Payable and PO data from SAP. These datasets
are stitched together and enriched with other datasets like HR org
hierarchy data (to see which org or team spent what) and geographical
data (to see which locations spend originated from).

One of the key enrichment of the spend data that is currently lacking is
the spend categorization. This mean, putting spend transactions into
right buckets based on what the spend type was. For example, a
transaction to buy computer hardward gets categorized under ``Hardware''
category, an invoice to pay Accenture for their services gets
categorized as ``Consulting Services'' and so on. This is a perfect
problem to solve with a classification model like KNN. We even did a
pilot with an Auto-ML software company and compared their model with the
one from our internal data scientists. Further details are given below:

\begin{itemize}
\tightlist
\item
  \textbf{Data size:} \textasciitilde{} 1M rows of invoice transactions
  per month
\item
  \textbf{Total features (columns) available:} 100+
\item
  \textbf{Response:} a category (from a defined hierarchy of spend
  categories) of spend assigned to each invoice line (which is
  represented by a single row of the data)
\item
  \textbf{Why is this important?} A key insight any procurement or
  finance team needs is details into where and how comapny's dollars are
  spend. If a procurement team can't even identify basic information
  like how much money the company spent on let's say `software', they
  can't do further time series and trend analysis on that category to
  manage it better. A properly classified spend data enables procurement
  teams to monitor the high spend categories and devise strategies to
  manage those categories better.
\item
  \textbf{Predictors Useful for the Model:}

  \begin{itemize}
  \tightlist
  \item
    \textbf{Supplier Name} - an unstructure data type predictor with
    supplier names entered as free text. E.g. Accenture might indicate
    spend is for consulting services whereas Infosys or Tata Consulting
    might indicate it's for managed service.
  \item
    \textbf{Business Unit}- a categorical data type with fixed names for
    different business units. Some BUs spend more of on one type of
    products or services. E.g. enterprise IT will almost always spend
    more on Hardware, Software, IT Services etc.
  \item
    \textbf{Finance Coding} (General Ledger Account Code) - a
    categorical data type with code numbers indicating the type of spend
    as defined by the Finance team.
  \item
    \textbf{Invoice Line Description} - an unstructured data type with
    free text explaining the what was purchased. There would most likely
    be hints in this text regarding what product or service was
    procured.
  \item
    \textbf{Supplier Location} - a categorical data type, location of a
    given supplier would provide useful information regarding the type
    of product or service procured. E.g. an invoice from Google in the
    Bay Area could be for any of the services they offer whereas an
    invoice from Google in Seattle would most likely be for Cloud
    Services only.
  \end{itemize}
\end{itemize}

\begin{center}\rule{0.5\linewidth}{\linethickness}\end{center}

\hypertarget{question-2.2}{%
\section{QUESTION 2.2}\label{question-2.2}}

\textbf{1. Using the support vector machine function ksvm contained in
the R package kernlab, find a good classifier for this data. Show the
equation of your classifier, and how well it classifies the data points
in the full data set. (Don't worry about test/validation data yet; we'll
cover that topic soon.)}

Getting the needed libraries.

\begin{Shaded}
\begin{Highlighting}[]
\KeywordTok{library}\NormalTok{(kernlab)}
\KeywordTok{library}\NormalTok{(kknn)}
\KeywordTok{library}\NormalTok{(caret)}
\KeywordTok{library}\NormalTok{(data.table)}
\end{Highlighting}
\end{Shaded}

Let's take a glimpse of the data. The original \textbf{654 rows} of the
data have been split into a trainig dataset (70\% of the rows) and
testing dataset (remaining 30\%). 70/30 split is selected after research
online and published work where this split has been recommneded the
most. It ensures enough data (70\%) for training and not 30\% sample is
also enough to check the validity of the model. 20\% for testing is also
mentioned but I went with 30\% so that there is enough new data for the
model to be properly vetted.

\begin{Shaded}
\begin{Highlighting}[]
\NormalTok{my_data <-}\StringTok{ }\KeywordTok{read.delim}\NormalTok{(}\StringTok{"data_2.2/credit_card_data-headers.txt"}\NormalTok{)}
\KeywordTok{nrow}\NormalTok{(my_data)}
\end{Highlighting}
\end{Shaded}

\begin{verbatim}
## [1] 654
\end{verbatim}

\begin{Shaded}
\begin{Highlighting}[]
\KeywordTok{set.seed}\NormalTok{(}\DecValTok{101}\NormalTok{) }\CommentTok{#this ensures that same datasets are reproduced in the future.}
\NormalTok{sample <-}\StringTok{ }\KeywordTok{sample.int}\NormalTok{(}\DataTypeTok{n =} \KeywordTok{nrow}\NormalTok{(my_data), }\DataTypeTok{size =} \KeywordTok{floor}\NormalTok{(.}\DecValTok{70}\OperatorTok{*}\KeywordTok{nrow}\NormalTok{(my_data)), }\DataTypeTok{replace =}\NormalTok{ F)}
\NormalTok{train_data <-}\StringTok{ }\NormalTok{my_data[sample,]}
\NormalTok{test_data  <-}\StringTok{ }\NormalTok{my_data[}\OperatorTok{-}\NormalTok{sample,]}

\KeywordTok{nrow}\NormalTok{(train_data)}
\end{Highlighting}
\end{Shaded}

\begin{verbatim}
## [1] 457
\end{verbatim}

\begin{Shaded}
\begin{Highlighting}[]
\KeywordTok{nrow}\NormalTok{(test_data)}
\end{Highlighting}
\end{Shaded}

\begin{verbatim}
## [1] 197
\end{verbatim}

\begin{Shaded}
\begin{Highlighting}[]
\KeywordTok{head}\NormalTok{(train_data)}
\end{Highlighting}
\end{Shaded}

\begin{verbatim}
##     A1    A2    A3    A8 A9 A10 A11 A12 A14 A15 R1
## 430  1 20.00 7.000 0.500  0   1   0   1   0   0  0
## 95   0 28.50 1.000 1.000  1   0   2   0 167 500  0
## 209  1 48.17 3.500 3.500  1   1   0   1 230   0  1
## 442  0 23.00 1.835 0.000  0   0   1   1 200  53  0
## 351  1 39.42 1.710 0.165  0   1   0   1 400   0  0
## 315  1 34.83 2.500 3.000  0   1   0   1 200   0  0
\end{verbatim}

I decided to use the KSVM function from Kernlab package instead of SVM
function from e1701 package. Main difference in KSVM are the different
kernal method available to try out non-linear classifications. e1701
only offers linear, radial basis , polynomial and sigmoid fittings,
whereas KSVM offers around 10 kernal functions including linear
(vanilladot), hyperbolic (tanhdot), ANOVA (anovadot) etc. This is
important b/c even though SVM theoretically finds the linear separation
my maximizing the margin, in real life, most available datasets have
non-linear decision boundaries. The kernal function of KSVM helps deals
with that.

Let's take a look at classifier from KSVM using linear kernel function
(vanilladot) and a relatively high C value (=100). The model is trained
and validated on the same training dataset as a starter.

\begin{Shaded}
\begin{Highlighting}[]
\CommentTok{#building model on training dataset}
\NormalTok{model_ksvm <-}\StringTok{ }\KeywordTok{ksvm}\NormalTok{( }\DataTypeTok{x =} \KeywordTok{as.matrix}\NormalTok{(train_data[,}\DecValTok{1}\OperatorTok{:}\DecValTok{10}\NormalTok{]),}
                    \DataTypeTok{y =} \KeywordTok{as.factor}\NormalTok{(train_data[,}\DecValTok{11}\NormalTok{]),}
                    \DataTypeTok{type=}\StringTok{'C-svc'}\NormalTok{,}
                    \DataTypeTok{kernel=} \StringTok{'vanilladot'}\NormalTok{,}
                    \DataTypeTok{C=}\DecValTok{100}\NormalTok{, }\DataTypeTok{scaled=}\OtherTok{TRUE}
\NormalTok{                    )}
\end{Highlighting}
\end{Shaded}

\begin{verbatim}
##  Setting default kernel parameters
\end{verbatim}

\begin{Shaded}
\begin{Highlighting}[]
\CommentTok{#checking model on the training data itself}
\NormalTok{predictions <-}\StringTok{ }\KeywordTok{predict}\NormalTok{(model_ksvm,}\DataTypeTok{newdata =}\NormalTok{ train_data[,}\DecValTok{1}\OperatorTok{:}\DecValTok{10}\NormalTok{])}
\KeywordTok{confusionMatrix}\NormalTok{(}\KeywordTok{factor}\NormalTok{(train_data[,}\DecValTok{11}\NormalTok{], }\DataTypeTok{levels =} \KeywordTok{c}\NormalTok{(}\DecValTok{1}\NormalTok{,}\DecValTok{0}\NormalTok{)),predictions)}
\end{Highlighting}
\end{Shaded}

\begin{verbatim}
## Confusion Matrix and Statistics
## 
##           Reference
## Prediction   0   1
##          0 192  55
##          1  13 197
##                                           
##                Accuracy : 0.8512          
##                  95% CI : (0.8152, 0.8826)
##     No Information Rate : 0.5514          
##     P-Value [Acc > NIR] : < 2.2e-16       
##                                           
##                   Kappa : 0.7049          
##                                           
##  Mcnemar's Test P-Value : 6.627e-07       
##                                           
##             Sensitivity : 0.9366          
##             Specificity : 0.7817          
##          Pos Pred Value : 0.7773          
##          Neg Pred Value : 0.9381          
##              Prevalence : 0.4486          
##          Detection Rate : 0.4201          
##    Detection Prevalence : 0.5405          
##       Balanced Accuracy : 0.8592          
##                                           
##        'Positive' Class : 0               
## 
\end{verbatim}

The model shows an accuracy of \textbf{85.12\%} on the training dataset.
Let's check model's performance on testing dataset.

\begin{Shaded}
\begin{Highlighting}[]
\CommentTok{#checking predictions on testing dataset}
\NormalTok{predictions <-}\StringTok{ }\KeywordTok{predict}\NormalTok{(model_ksvm,}\DataTypeTok{newdata =}\NormalTok{ test_data[,}\DecValTok{1}\OperatorTok{:}\DecValTok{10}\NormalTok{])}
\KeywordTok{confusionMatrix}\NormalTok{(}\KeywordTok{factor}\NormalTok{(test_data[,}\DecValTok{11}\NormalTok{], }\DataTypeTok{levels =} \KeywordTok{c}\NormalTok{(}\DecValTok{1}\NormalTok{,}\DecValTok{0}\NormalTok{)),predictions)}
\end{Highlighting}
\end{Shaded}

\begin{verbatim}
## Confusion Matrix and Statistics
## 
##           Reference
## Prediction  0  1
##          0 94 17
##          1  4 82
##                                           
##                Accuracy : 0.8934          
##                  95% CI : (0.8417, 0.9328)
##     No Information Rate : 0.5025          
##     P-Value [Acc > NIR] : < 2.2e-16       
##                                           
##                   Kappa : 0.7869          
##                                           
##  Mcnemar's Test P-Value : 0.008829        
##                                           
##             Sensitivity : 0.9592          
##             Specificity : 0.8283          
##          Pos Pred Value : 0.8468          
##          Neg Pred Value : 0.9535          
##              Prevalence : 0.4975          
##          Detection Rate : 0.4772          
##    Detection Prevalence : 0.5635          
##       Balanced Accuracy : 0.8937          
##                                           
##        'Positive' Class : 0               
## 
\end{verbatim}

The model's accuracy goes up to \textbf{89.34\%} on testing dataset. The
accuracy scores are not expected to match in general across training \&
testing datasets. This is expected because because sometimes small
overfitting of the training dataset is inevitable, making the model's
performance on training data better, whereas other times the training
dataset performs better indicating the model is truly working well.

Since I need to find the best C value for the mode, I will run a loop
for different C values and document the accuracy to see which C value
performs the best. 20 C values separated by 1e+/-01 are used in the
loop.

\begin{Shaded}
\begin{Highlighting}[]
\NormalTok{TestCs <-}\StringTok{ }\KeywordTok{c}\NormalTok{(}\DecValTok{10}\OperatorTok{**}\NormalTok{(}\OperatorTok{-}\DecValTok{10}\OperatorTok{:}\DecValTok{10}\NormalTok{)) }

\NormalTok{iter =}\StringTok{ }\KeywordTok{length}\NormalTok{(TestCs)}

\NormalTok{results <-}\StringTok{ }\KeywordTok{matrix}\NormalTok{(}\OtherTok{NA}\NormalTok{, }\DataTypeTok{nrow=}\NormalTok{iter, }\DataTypeTok{ncol=}\DecValTok{2}\NormalTok{)}
\KeywordTok{colnames}\NormalTok{(results) <-}\StringTok{ }\KeywordTok{c}\NormalTok{(}\StringTok{"Cvalue"}\NormalTok{,}\StringTok{"Accuracy"}\NormalTok{)}

\CommentTok{#looping Cs on the model and documenting accuracy percentage}
\ControlFlowTok{for}\NormalTok{(i }\ControlFlowTok{in} \DecValTok{1}\OperatorTok{:}\NormalTok{iter)\{}
\NormalTok{    model_ksvm <-}\StringTok{ }\KeywordTok{ksvm}\NormalTok{( }\DataTypeTok{x =} \KeywordTok{as.matrix}\NormalTok{(train_data[,}\DecValTok{1}\OperatorTok{:}\DecValTok{10}\NormalTok{]),}
                    \DataTypeTok{y =} \KeywordTok{as.factor}\NormalTok{(train_data[,}\DecValTok{11}\NormalTok{]),}
                    \DataTypeTok{type=}\StringTok{'C-svc'}\NormalTok{,}
                    \DataTypeTok{kernel=} \StringTok{'vanilladot'}\NormalTok{,}
                    \DataTypeTok{C=}\NormalTok{TestCs[[i]], }
                    \DataTypeTok{scaled=}\OtherTok{TRUE}
\NormalTok{                    )}
\NormalTok{    predictions <-}\StringTok{ }\KeywordTok{predict}\NormalTok{(model_ksvm,}\DataTypeTok{newdata =}\NormalTok{ test_data[,}\DecValTok{1}\OperatorTok{:}\DecValTok{10}\NormalTok{])}
\NormalTok{    results[i,] <-}\StringTok{ }\KeywordTok{c}\NormalTok{(TestCs[[i]], }\KeywordTok{sum}\NormalTok{(predictions }\OperatorTok{==}\StringTok{ }\NormalTok{test_data[,}\DecValTok{11}\NormalTok{]) }\OperatorTok{/}\StringTok{ }\KeywordTok{nrow}\NormalTok{(test_data))}
\NormalTok{\}}
\end{Highlighting}
\end{Shaded}

\begin{verbatim}
##  Setting default kernel parameters  
##  Setting default kernel parameters  
##  Setting default kernel parameters  
##  Setting default kernel parameters  
##  Setting default kernel parameters  
##  Setting default kernel parameters  
##  Setting default kernel parameters  
##  Setting default kernel parameters  
##  Setting default kernel parameters  
##  Setting default kernel parameters  
##  Setting default kernel parameters  
##  Setting default kernel parameters  
##  Setting default kernel parameters  
##  Setting default kernel parameters  
##  Setting default kernel parameters  
##  Setting default kernel parameters  
##  Setting default kernel parameters  
##  Setting default kernel parameters  
##  Setting default kernel parameters  
##  Setting default kernel parameters  
##  Setting default kernel parameters
\end{verbatim}

\begin{Shaded}
\begin{Highlighting}[]
\NormalTok{results}
\end{Highlighting}
\end{Shaded}

\begin{verbatim}
##       Cvalue  Accuracy
##  [1,]  1e-10 0.5634518
##  [2,]  1e-09 0.5634518
##  [3,]  1e-08 0.5634518
##  [4,]  1e-07 0.5634518
##  [5,]  1e-06 0.5634518
##  [6,]  1e-05 0.5634518
##  [7,]  1e-04 0.5634518
##  [8,]  1e-03 0.8324873
##  [9,]  1e-02 0.8934010
## [10,]  1e-01 0.8934010
## [11,]  1e+00 0.8934010
## [12,]  1e+01 0.8934010
## [13,]  1e+02 0.8934010
## [14,]  1e+03 0.8934010
## [15,]  1e+04 0.8934010
## [16,]  1e+05 0.8934010
## [17,]  1e+06 0.8781726
## [18,]  1e+07 0.7106599
## [19,]  1e+08 0.6903553
## [20,]  1e+09 0.6700508
## [21,]  1e+10 0.6700508
\end{verbatim}

The results of the loop above show that for C values b/w 0.01 to 100,000
the accuracy of the model does not change i.e.89.34\%. Accuracy drops
below 0.01 and above 100,000.

I will use \textbf{C = 100} value to find the classifier from KSVM
model. This C value is in the middle of the range for which accuracy is
the same. This ensures that there is neither too much emphasis on the
margin nor the total error. At the same time, it also indicates the
right trade-off b/w margin and total error because for a higher than
100,000 C and lower than 0.01 C, accuracy is lower.

\begin{Shaded}
\begin{Highlighting}[]
\CommentTok{#final classifer using C=100}

\NormalTok{final_model_ksvm <-}\StringTok{ }\KeywordTok{ksvm}\NormalTok{( }\DataTypeTok{x =} \KeywordTok{as.matrix}\NormalTok{(train_data[,}\DecValTok{1}\OperatorTok{:}\DecValTok{10}\NormalTok{]),}
                    \DataTypeTok{y =} \KeywordTok{as.factor}\NormalTok{(train_data[,}\DecValTok{11}\NormalTok{]),}
                    \DataTypeTok{type=}\StringTok{'C-svc'}\NormalTok{,}
                    \DataTypeTok{kernel=} \StringTok{'vanilladot'}\NormalTok{,}
                    \DataTypeTok{C=}\DecValTok{100}\NormalTok{, }\DataTypeTok{scaled=}\OtherTok{TRUE}
\NormalTok{                    ) }
\end{Highlighting}
\end{Shaded}

\begin{verbatim}
##  Setting default kernel parameters
\end{verbatim}

\begin{Shaded}
\begin{Highlighting}[]
\NormalTok{final_predictions <-}\StringTok{ }\KeywordTok{predict}\NormalTok{(final_model_ksvm,}\DataTypeTok{newdata =}\NormalTok{ test_data[,}\DecValTok{1}\OperatorTok{:}\DecValTok{10}\NormalTok{])}
\KeywordTok{confusionMatrix}\NormalTok{(}\KeywordTok{factor}\NormalTok{(test_data[,}\DecValTok{11}\NormalTok{], }\DataTypeTok{levels =} \KeywordTok{c}\NormalTok{(}\DecValTok{1}\NormalTok{,}\DecValTok{0}\NormalTok{)),final_predictions)}
\end{Highlighting}
\end{Shaded}

\begin{verbatim}
## Confusion Matrix and Statistics
## 
##           Reference
## Prediction  0  1
##          0 94 17
##          1  4 82
##                                           
##                Accuracy : 0.8934          
##                  95% CI : (0.8417, 0.9328)
##     No Information Rate : 0.5025          
##     P-Value [Acc > NIR] : < 2.2e-16       
##                                           
##                   Kappa : 0.7869          
##                                           
##  Mcnemar's Test P-Value : 0.008829        
##                                           
##             Sensitivity : 0.9592          
##             Specificity : 0.8283          
##          Pos Pred Value : 0.8468          
##          Neg Pred Value : 0.9535          
##              Prevalence : 0.4975          
##          Detection Rate : 0.4772          
##    Detection Prevalence : 0.5635          
##       Balanced Accuracy : 0.8937          
##                                           
##        'Positive' Class : 0               
## 
\end{verbatim}

Using this model, let's find the coefficients for the equation.

\begin{Shaded}
\begin{Highlighting}[]
\CommentTok{#calculting a1...am}
\NormalTok{a <-}\StringTok{ }\KeywordTok{colSums}\NormalTok{(final_model_ksvm}\OperatorTok{@}\NormalTok{xmatrix[[}\DecValTok{1}\NormalTok{]] }\OperatorTok{*}\StringTok{ }\NormalTok{final_model_ksvm}\OperatorTok{@}\NormalTok{coef[[}\DecValTok{1}\NormalTok{]])}
\NormalTok{a}
\end{Highlighting}
\end{Shaded}

\begin{verbatim}
##            A1            A2            A3            A8            A9 
## -0.0009206098 -0.0044367594 -0.0026209194  0.0047287318  1.0034635331 
##           A10           A11           A12           A14           A15 
## -0.0021271935 -0.0001701517 -0.0003920085 -0.0029128439  0.1051406257
\end{verbatim}

\begin{Shaded}
\begin{Highlighting}[]
\CommentTok{#calculting a0}
\NormalTok{a0 <-}\StringTok{ }\OperatorTok{-}\NormalTok{final_model_ksvm}\OperatorTok{@}\NormalTok{b}
\NormalTok{a0}
\end{Highlighting}
\end{Shaded}

\begin{verbatim}
## [1] 0.1075138
\end{verbatim}

Thus, the final equation for every point j would be:

\(-0.0009206098x_{A1j} - 0.0044367594x_{A2j} - 0.0026209194x_{A3j} + 0.0047287318x_{A8j} + 1.0034635331x_{A9j} - 0.0021271935x_{A10j}\)
\(- 0.0001701517x_{A11j} - 0.0003920085x_{A12j} - 0.0029128439x_{A14j} + 0.1051406257x_{A15j} + 0.1075138 = 0\)

\textbf{2. You are welcome, but not required, to try other (nonlinear)
kernels as well; we're not covering them in this course, but they can
sometimes be useful and might provide better predictions than
vanilladot.}

In order to try the other kernels, let's loop over the model and use
other available kernels. I will however skip the `stringdot' kernel
because it specializes operating on strings and our data does not have
strings. Though I had landed on C=100 based on the analysis above, I
will however try a few C values from the range for which accuracy of
linear kernel was the same.

\begin{Shaded}
\begin{Highlighting}[]
\NormalTok{kernels <-}\StringTok{ }\KeywordTok{c}\NormalTok{(}\StringTok{'rbfdot'}\NormalTok{,}\StringTok{'polydot'}\NormalTok{,}\StringTok{'tanhdot'}\NormalTok{,}\StringTok{'laplacedot'}\NormalTok{,}\StringTok{'besseldot'}\NormalTok{,}\StringTok{'anovadot'}\NormalTok{,}\StringTok{'splinedot'}\NormalTok{) }

\NormalTok{iter =}\StringTok{ }\KeywordTok{length}\NormalTok{(kernels)}


\NormalTok{kernel_results <-}\StringTok{ }\KeywordTok{matrix}\NormalTok{(}\OtherTok{NA}\NormalTok{, }\DataTypeTok{nrow=}\NormalTok{iter, }\DataTypeTok{ncol=}\DecValTok{5}\NormalTok{)}
\KeywordTok{colnames}\NormalTok{(kernel_results) <-}\StringTok{ }\KeywordTok{c}\NormalTok{(}\StringTok{"Kernel"}\NormalTok{,}\StringTok{"C=0.01"}\NormalTok{,}\StringTok{"C=1.00"}\NormalTok{,}\StringTok{"C=100"}\NormalTok{,}\StringTok{"C=10000"}\NormalTok{)}

\CommentTok{#looping different kernel on the model and documenting accuracy percentage for 4 different C values}
\ControlFlowTok{for}\NormalTok{(i }\ControlFlowTok{in} \DecValTok{1}\OperatorTok{:}\NormalTok{iter)\{}
\NormalTok{    c0}\FloatTok{.01}\NormalTok{ <-}\StringTok{ }\KeywordTok{ksvm}\NormalTok{( }\DataTypeTok{x =} \KeywordTok{as.matrix}\NormalTok{(train_data[,}\DecValTok{1}\OperatorTok{:}\DecValTok{10}\NormalTok{]),}
                    \DataTypeTok{y =} \KeywordTok{as.factor}\NormalTok{(train_data[,}\DecValTok{11}\NormalTok{]),}
                    \DataTypeTok{type=}\StringTok{'C-svc'}\NormalTok{,}
                    \DataTypeTok{kernel=}\NormalTok{ kernels[[i]],}
                    \DataTypeTok{C=}\FloatTok{0.01}\NormalTok{, }
                    \DataTypeTok{scaled=}\OtherTok{TRUE}
\NormalTok{                    )}
\NormalTok{    c1}\FloatTok{.00}\NormalTok{ <-}\StringTok{ }\KeywordTok{ksvm}\NormalTok{( }\DataTypeTok{x =} \KeywordTok{as.matrix}\NormalTok{(train_data[,}\DecValTok{1}\OperatorTok{:}\DecValTok{10}\NormalTok{]),}
                    \DataTypeTok{y =} \KeywordTok{as.factor}\NormalTok{(train_data[,}\DecValTok{11}\NormalTok{]),}
                    \DataTypeTok{type=}\StringTok{'C-svc'}\NormalTok{,}
                    \DataTypeTok{kernel=}\NormalTok{ kernels[[i]],}
                    \DataTypeTok{C=}\DecValTok{1}\NormalTok{, }
                    \DataTypeTok{scaled=}\OtherTok{TRUE}
\NormalTok{                    )}
\NormalTok{    c100 <-}\StringTok{ }\KeywordTok{ksvm}\NormalTok{( }\DataTypeTok{x =} \KeywordTok{as.matrix}\NormalTok{(train_data[,}\DecValTok{1}\OperatorTok{:}\DecValTok{10}\NormalTok{]),}
                    \DataTypeTok{y =} \KeywordTok{as.factor}\NormalTok{(train_data[,}\DecValTok{11}\NormalTok{]),}
                    \DataTypeTok{type=}\StringTok{'C-svc'}\NormalTok{,}
                    \DataTypeTok{kernel=}\NormalTok{ kernels[[i]],}
                    \DataTypeTok{C=}\DecValTok{100}\NormalTok{, }
                    \DataTypeTok{scaled=}\OtherTok{TRUE}
\NormalTok{                    )}
\NormalTok{    c10000 <-}\StringTok{ }\KeywordTok{ksvm}\NormalTok{( }\DataTypeTok{x =} \KeywordTok{as.matrix}\NormalTok{(train_data[,}\DecValTok{1}\OperatorTok{:}\DecValTok{10}\NormalTok{]),}
                    \DataTypeTok{y =} \KeywordTok{as.factor}\NormalTok{(train_data[,}\DecValTok{11}\NormalTok{]),}
                    \DataTypeTok{type=}\StringTok{'C-svc'}\NormalTok{,}
                    \DataTypeTok{kernel=}\NormalTok{ kernels[[i]],}
                    \DataTypeTok{C=}\DecValTok{10000}\NormalTok{, }
                    \DataTypeTok{scaled=}\OtherTok{TRUE}
\NormalTok{                    )}
\NormalTok{    predc0}\FloatTok{.01}\NormalTok{ <-}\StringTok{ }\KeywordTok{predict}\NormalTok{(c0}\FloatTok{.01}\NormalTok{,}\DataTypeTok{newdata =}\NormalTok{ test_data[,}\DecValTok{1}\OperatorTok{:}\DecValTok{10}\NormalTok{])}
\NormalTok{    predc1}\FloatTok{.00}\NormalTok{ <-}\StringTok{ }\KeywordTok{predict}\NormalTok{(c1}\FloatTok{.00}\NormalTok{,}\DataTypeTok{newdata =}\NormalTok{ test_data[,}\DecValTok{1}\OperatorTok{:}\DecValTok{10}\NormalTok{])}
\NormalTok{    predc100 <-}\StringTok{ }\KeywordTok{predict}\NormalTok{(c100,}\DataTypeTok{newdata =}\NormalTok{ test_data[,}\DecValTok{1}\OperatorTok{:}\DecValTok{10}\NormalTok{])}
\NormalTok{    predc10000 <-}\StringTok{ }\KeywordTok{predict}\NormalTok{(c10000,}\DataTypeTok{newdata =}\NormalTok{ test_data[,}\DecValTok{1}\OperatorTok{:}\DecValTok{10}\NormalTok{])}
\NormalTok{    kernel_results[i,] <-}\StringTok{ }\KeywordTok{c}\NormalTok{(kernels[[i]], }
                            \KeywordTok{sum}\NormalTok{(predc0}\FloatTok{.01} \OperatorTok{==}\StringTok{ }\NormalTok{test_data[,}\DecValTok{11}\NormalTok{]) }\OperatorTok{/}\StringTok{ }\KeywordTok{nrow}\NormalTok{(test_data),}
                            \KeywordTok{sum}\NormalTok{(predc1}\FloatTok{.00} \OperatorTok{==}\StringTok{ }\NormalTok{test_data[,}\DecValTok{11}\NormalTok{]) }\OperatorTok{/}\StringTok{ }\KeywordTok{nrow}\NormalTok{(test_data),}
                            \KeywordTok{sum}\NormalTok{(predc100 }\OperatorTok{==}\StringTok{ }\NormalTok{test_data[,}\DecValTok{11}\NormalTok{]) }\OperatorTok{/}\StringTok{ }\KeywordTok{nrow}\NormalTok{(test_data),}
                            \KeywordTok{sum}\NormalTok{(predc10000 }\OperatorTok{==}\StringTok{ }\NormalTok{test_data[,}\DecValTok{11}\NormalTok{]) }\OperatorTok{/}\StringTok{ }\KeywordTok{nrow}\NormalTok{(test_data)}
\NormalTok{                            )}
\NormalTok{\}}
\end{Highlighting}
\end{Shaded}

\begin{verbatim}
##  Setting default kernel parameters  
##  Setting default kernel parameters  
##  Setting default kernel parameters  
##  Setting default kernel parameters  
##  Setting default kernel parameters  
##  Setting default kernel parameters  
##  Setting default kernel parameters  
##  Setting default kernel parameters  
##  Setting default kernel parameters  
##  Setting default kernel parameters  
##  Setting default kernel parameters  
##  Setting default kernel parameters  
##  Setting default kernel parameters  
##  Setting default kernel parameters  
##  Setting default kernel parameters  
##  Setting default kernel parameters  
##  Setting default kernel parameters  
##  Setting default kernel parameters  
##  Setting default kernel parameters  
##  Setting default kernel parameters
\end{verbatim}

\begin{Shaded}
\begin{Highlighting}[]
\NormalTok{kernel_results}
\end{Highlighting}
\end{Shaded}

\begin{verbatim}
##      Kernel       C=0.01              C=1.00              C=100              
## [1,] "rbfdot"     "0.563451776649746" "0.883248730964467" "0.82741116751269" 
## [2,] "polydot"    "0.893401015228426" "0.893401015228426" "0.893401015228426"
## [3,] "tanhdot"    "0.888324873096447" "0.705583756345178" "0.796954314720812"
## [4,] "laplacedot" "0.563451776649746" "0.893401015228426" "0.878172588832487"
## [5,] "besseldot"  "0.563451776649746" "0.873096446700508" "0.776649746192893"
## [6,] "anovadot"   "0.893401015228426" "0.893401015228426" "0.888324873096447"
## [7,] "splinedot"  "0.878172588832487" "0.812182741116751" "0.791878172588833"
##      C=10000            
## [1,] "0.796954314720812"
## [2,] "0.893401015228426"
## [3,] "0.786802030456853"
## [4,] "0.878172588832487"
## [5,] "0.751269035532995"
## [6,] "0.873096446700508"
## [7,] "0.791878172588833"
\end{verbatim}

The output of the analysis above shows that accuracy is varying quite a
bit for some kernels over C values but for all kernels and C value
iterations, I do not see any accuracy over 89.34\% which was our final
accuracy from C=100 linear model above. This means, even though there
are non-linear models that are closer to the linear model's accuracy,
none of them were better.

Among the seven kernel tried, polydot (Polynomial kernel-looks at not
just input features but their combinations as well),laplacedot (which is
similarto Gaussian \& exponential kernels which are radial basis
functions) and ANNOVAdot (also a radial basis function) came close to
linear model's 89\% accuracy mark.

\textbf{3. Using the k-nearest-neighbors classification function kknn
contained in the R kknn package, suggest a good value of k, and show how
well it classifies that data points in the full data set. Don't forget
to scale the data (scale=TRUE in kknn).}

\end{document}
